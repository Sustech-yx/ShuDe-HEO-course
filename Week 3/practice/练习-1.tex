\documentclass[a4paper,10pt]{article}
\usepackage{listings}
\title{A Simple Quiz on Hand-Written Code}
\begin{document}

\maketitle
\begin{center}
\large{Berland Crosswordtime}\\
\small
limit per test: 2 seconds\\
memory limit per test: 256 megabytes\\
input: standard input\\
output: standard output\\
\end{center}
Berland crossword is a puzzle that is solved on a square grid with $n$ rows and $n$ columns. Initially all the cells are white.\\
To solve the puzzle one has to color some cells on the border of the grid black in such a way that:
\begin{itemize}
\item exactly $U$ cells in the top row are black;
\item exactly $R$ cells in the rightmost column are black;
\item exactly $D$ cells in the bottom row are black;
\item exactly $L$ cells in the leftmost column are black.
\end{itemize}
Note that you can color zero cells black and leave every cell white.\\
Your task is to check if there exists a solution to the given puzzle.
InputThe first line contains a single integer $t$ ($1 \le t \le 1000$) — the number of testcases.
Then the descriptions of $t$ testcases follow.
The only line of each testcase contains $5$ integers $n, U, R, D, L$ ($2 \le n \le 100$; $0 \le U, R, D, L \le n$).
OutputFor each testcase print "YES" if the solution exists and "NO" otherwise.
You may print every letter in any case you want (so, for example, the strings yEs, yes, Yes and YES are all recognized as positive answer).\\
Sample Input:\\
4\\
5 2 5 3 1\\
3 0 0 0 0\\
4 4 1 4 0\\
2 1 1 1 1\\
Sample Output:\\
YES\\
YES\\
NO\\
YES\\

\[Link\](https://codeforces.com/contest/1494/problem/B)
\newpage
\begin{lstlisting}
public class Main {
    static Scanner in = new Scanner(System.in);
    static PrintWriter out = new PrintWriter(new OutputStreamWriter(System.out));
    public static void main(String[] args) {
        int t = in.nextInt();
        while (t -- > 0) {
            solve();
        }
        out.close();
    }
    static void solve() {
        int n = in.nextInt();
        int U = in.nextInt();
        int R = in.nextInt();
        int D = in.nextInt();
        int L = in.nextInt();
        int u, r, d, l;
        for (int i = 0; i < 16; i++) {
            u = U - ((i >> 0) &1) - ((i >> 3) &1);
            r = R - ((i >> 0) &1) - ((i >> 1) &1);
            d = D - ((i >> 1) &1) - ((i >> 2) &1);
            l = L - ((i >> 2) &1) - ((i >> 3) &1);
            if (
                    u >= 0 && u <= n-2 &&
                    l >= 0 && l <= n-2 &&
                    d >= 0 && d <= n-2 &&
                    r >= 0 && r <= n-2
            ) {out.println("YES"); return;}
        }
        out.println("NO");
    }
}
\end{lstlisting}




\end{document}
\end{document}
